

\section{Fermi surface nesting as a pairing mechanism}

The charge carrier in a superconducting condensate is a Cooper pair - a quasi-particle comprising of a bound state of two electrons or two holes with opposite spin and momentum. Evidence for this configuration arises as a natural result of the Ginzberg-Landau model which, when applied to a superconducting system, gives the charge of the quasi-particle carriers as $2e$, where $e$ is the charge of an electron. Given that due to their like charges two free electrons repel, it is natural to ask what could overcome the electromagnetic force to cause these electrons to remain bound in this quasi-particle state.

Bardeen, Cooper and Schreiffer established much of the theoretical basis --- from which the Ginzberg--Landau model can be derived --- in \textit{BCS theory} (named after the authors) and within the framework of BCS theory, wrote a 1957 paper\cite{Bardeen1957} detailed a pairing mechanism known as the \textit{BCS model} which would explain how these electron remained bound together. The model is based around the concept of phonons scattering off ions which well suited the superconducting materials known at the time. Phenomenologically, the mechanism of attraction is straightforward. Electrons moving through a crystal lattice attract ions on the lattice sites. These heavy ions respond slowly and are drawn in \textit{behind} the electron. This has the effect of both screening the negative electron charge as well as providing an attractive positive potential for any electron following the original electron. The net effect is the leading electron draws the following electron in its wake, thus coupling them with one another. The wavelike distortion of the ions in the lattice can be considered as a phonon, and the interaction between the electrons and the lattice can be modelled as electron--phonon--electron scattering.

The BCS model on top of BCS theory accurately describes what we now know as \textit{conventional superconductivity}, that is pairing which forms a spin-singlet state ($S=0$) and which has zero orbital angular momentum ($L=0$). It was not until the discovery of superfluidity\footnote{Superfluidity and superconductivity share much of the same physics although rather than electrons or holes pairing, molecules pair instead. Parallels betwen the two are discussed in ref.\cite{Annett2010}} in $^3$He in 1972\cite{Osheroff1972} that it became apparent that there may exist forms of pairing that resulted in spin-triplet pairing state ($S=1$) with $L>0$. This was later confirmed when superconducting analogues were found in the form of heavy Fermion materials. What really spurred the explosion in interest though was the 1986 discovery by Bednorz and M\"uller\cite{Bednorz} of high transition temperature (\Tc) superconductivity in the cuprates and, more recently, the `pnictides' by Kamihara et al.\cite{Kamihara2008}. The cuprate class of materials that Bednorz and M\"uller found to be superconducting have \Tc~s far in excess of any previously known superconducting materials and although the BCS model phonon pairing may play a part, the predominant pairing mechanism in the \highTc materials is likely to be something else entirely.

\subsection{The case against conventional superconductivity in \highTc}

There is a great deal of evidence in the literature for non-BCS model pairing in the \highTc and heavy Fermion materials. Although the pairing wavefunction cannot be measured directly with current techniques, experiments indirectly infer \textit{unconventional} i.e. non s-wave, BCS-model, characteristics. For example, analysis on penetration depth measurements of YBa$_2$Cu$_3$O$_{7-\delta}$ show power law behaviour\cite{Annett1991}, indicating that there exists states within the momentum averaged gap. SQUID measurements and Josephson tunneling experiments on the same material have confirmed alternating phase of the condensate wavefunction which points strongly to \DxTwoyTwo--wave symmetry\cite{VanHarlingen1994} (see also refs. therein). As for other cuprate materials, specific heat measurements on \BSCO\cite{Wang2011}, as well as peentration depth measurements on LSCO\cite{Froehlich1996} have also proved consistant with $d$-wave pairing. 

More evidence against conventional superconductivity include the unusual normal state (i.e. non-superconducting) state properties of the cuprates and heavy Fermion materials. The BCS model is grounded in Landau Fermi liquid theory which models interacting itinerent electrons with quasiparticles of heavier effective mass than ordinary electrons and holes. A hallmark of Fermi liquid behaviour is a $T^2$ dependence of the resistance, however experiments on the cuprate La$_{2-x}$Sr$_{x}$CuO$_4$\cite{Cooper2009} and a heavy Fermion material\cite{Custers2003} have demonstrated fractional power law behaviour, $T^\gamma$ where $1 < \gamma < 2$, at temepratures above the superconducting transition. Given that the Fermi liquid model breaks down in these examples, it follows that the BCS-model also is likely on shaky ground for these materials.

There are several arguments against phonons as the sole pairing mechanism in the pnictide case, Boeri et al.\cite{Boeri2008} and Mazin et al.\cite{Mazin2008} present calculations showing that the magnitude of the phonon pairing strength is not adequate for the high \Tc values attained in LaAsOF, Haule et al.\cite{Haule2008} note in the same material that the gradient of the density of states (DOS) at the Fermi level is such that you would expect an increase in DOS and hence \Tc with hole doping if the BCS model held, however the reverse is true. Non Fermi-liquid behaviour was demonstrated in the \BaFePAs series\cite{Jiang2009,Kasahara2010} and evidence for nodes in the gap function have been found in LaFePO\cite{Fletcher2009} and the \BaFePAs series\cite{Zhang2011,Yamashita2011a,Suzuki2011} although not in % TODO

It is interesting to note that Unlike the cuprates which universally show a \DxTwoyTwo gap symmetry, the pnictide materials are note all alike. As a result, it may prove that the nature of the supercondcutivity may not be universal amongst the pnictide materials. Irrespective of this, there is no evidence for BCS model pairing in the pnictide materials and in many cases, BCS pairing has been shown to be insufficient.


\subsection{Spin-fluctuations}

Soon after the discovery of the pnictide materials, a possible pairing mechanism was proposed based on spin density wave fluctuations. The original paper suggested a $s_{\pm}$ gap symmetry which does not feature any nodes however 

\subsection{Pnictides}

Some arguments against the BCS theory of pairing \cite{Haule2008,Yndurain2009,Mazin2008} based on arguments of 

FS nesting not the only cause of spin-fluctuations, also can be caused by frustrated superexchange for example % TODO

Spin fluctuations mediate a repulsive interaction between Cooper pair candidates.

The anisotropic BCS equations specify that repulsive coupling between carriers can be pairing provided the order parameter changes sign over the coupling vector.


There are several proposed mechanisms presently on offer including charge fluctuations resulting in large ion polarisation \cite{Berciu2009}, however this was contested by Mazin and Schmalian\cite{Mazin2009}.


Of these theories, the one with arguably the most traction at present is that of spin-fluctuation mediated pairing. 

TODO: What actually is the cause of the attraction in the nesting picture? ... Spin fluctuation intereaction in real space is approximately propoprtional to the dipole interaction $V=-\mu . \mu \chi(r)$\cite{Bergemann2003} 

Strong correlations - the interaction energy is much greater than the kinetic energy for the states
When correlations present, Cooper pairs are assumed to be pairs of Landau quasiparticles


% The Stoner condition of $\mathcal{N}_0 I > 1$ -- where $\mathcal{N}_0$ is the density of states at the Fermi energy and $I$ is the molecular field constant, that scales the magnetism given a field -- indicates an energy instability\cite{Kubler2000}
