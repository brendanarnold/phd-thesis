
\section{The cuprate phase diagram}

Contrary to the recently discovered pnictide materials, the phase diagrams for the cuprates are very similar with principle differences being only in scale. However this universality comes with an greater abundance of features which are not present or at least have not yet been charactersised in the pnictides. Figure~\ref{Fig:Intro:PhaseDiagrams} shows the phase diagram for \TODO{Give a particular example that includes electron doping} in particular but the features are shared between the cuprates. Looking more closely at the hole doping side, which in general is where the superconductivity is more robust, we start in a Mott insulating antiferromagnetic \ac{SDW} state in the undoped parent compound, this gradually dissapates as doping is introduced along the temeprature scale, T$_N$



\section{Mott physics}

The Hubbard model takes the relatively simple and solvable tight-binding model and introduces an Anderson term which raises the energy for double occupancy by an amount $U$, known as the `Hubbard U'. This simple change deeply enriches the physics with one of the outcomes being the existance of the Mott insulating state which occurs when each lattice site is half filled with a single electron. The energy cost for an electron to hop to an adjacent site is so high that it locks the electrons in place, preventing effective conduction. Introducing holes (or electrons) allows once again hopping to take place and the eigenstates are no longer entireley localised.
