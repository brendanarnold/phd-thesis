
\section{The cuprate phase diagram}

The phase diagrams for the \highTc materials show a remarkable consistancy across the cuprates\footnote{This is in constrast with the recently discovered pnictide materials which show significant variations in scalings and even composition}. However this universality amongst the cuprates comes with an abundance of features which provide for some complex physical interactions and fragile intermediate phases. 

One of the most contraversial regions of the phase diagram is the so called pseudogap phase. This is a regions which has demonstrated in many spectroscopic probes such as \ac{STM} and \ac{ARPES} the existance of an energy gap and the possibility of Cooper pairs as evidenced by Nernst measurements. However 

Figure~\ref{Fig:Intro:PhaseDiagrams} shows the phase diagram for \TODO{Give a particular example that includes electron doping} in particular that demonstrates both electron and hole doped progressions. Looking more closely at the hole doping side, which in general is where the superconductivity is more robust, we start in a Mott insulating antiferromagnetic \ac{SDW} state in the undoped parent compound, this gradually dissapates as doping is introduced along the temeprature scale, T$_N$



\section{Mott physics}

The Hubbard model takes the relatively simple and solvable tight-binding model and introduces an Anderson term which raises the energy for double occupancy by an amount $U$, known as the `Hubbard U'. This simple change deeply enriches the physics with one of the outcomes being the existance of the Mott insulating state which occurs when each lattice site is half filled with a single electron. The energy cost for an electron to hop to an adjacent site is so high that it locks the electrons in place, preventing effective conduction. Introducing holes (or electrons) allows once again hopping to take place and the eigenstates are no longer entireley localised.
