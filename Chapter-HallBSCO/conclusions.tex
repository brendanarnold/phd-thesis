
\section{Conclusions}

High quality crystals of Pb and Sr doped \ac{BSCO} were sourced and studied in the normal state by high-field magnetotransport measurements down to low temperatures, thereby determining the low temperature Hall behaviour. The samples exhibited a sharp change in in the $R_H(\unit{0}{\kelvin})/R_H(\unit{300}{\kelvin})$ indicating a \TODO{Ask Nigel for suggestions as to what this might mean}.

The data was modeled using a simple form of the Ong construction and was found to fit the relative scattering rates in the underdoped regime reasonably well. This suggests that with further refinement it could be used to explain the physics at underdoped side of the phase diagram without resorting to complex Fermi surface reconstruction scenarios proposed by LeBeouf \etal. The first port of call for the refinement would be the inclusion of the Fermi velocity in the scattering rate which may also improve the agreement in the overdoped side.

A novel doping determination technique is presented based on the method outlined by Ando \etal but comparing the \ac{BSCO} samples to the recently determined doping in overdoped \ac{TL2201} using \ac{dHvA}. The method assigns doping values that fall between the `universal' method of Presland/Tallon and those found from \ac{ARPES} measurements by Kondo \etal

A natural continuation of this work would include a more precise determination of the low field region to determine with more certainty if the low temperature behaviour is truly $T$-linear or it plateaus at very low temperatures as found by Balkirev \etal in underdoped samples~\cite{Balakirev2003} and then attempt to model it using the Ong construction.
