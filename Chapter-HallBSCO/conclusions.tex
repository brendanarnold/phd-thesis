
\section{Conclusions}

High quality crystals of Pb and Sr doped \ac{BSCO} were sourced and studied in the normal state by magnetotransport measurements down to low temperatures, thereby determining the low temperature Hall behaviour. We found some evidence that determining the doping in \ac{BSCO} by normalising to a hybrid combination of the Tallon relation for optimal and underdoped samples and to the \ac{TL2201} data for overdoped samples gives dopings which are more consistent with the literature \TODO{Include a comparison with LSCO, not other BSCO samples scaled to Tallon!}

Two possible scenarios could explain the Hall behaviour in these \ac{BSCO} measurements. The first is the proximity of the van-Hove singularity, the second is an anisotropic scattering rate. The evidence for anisotropy could be explained in the former case by an anisotropy in the saddle point flat regions of the bandstructure, in the latter case the explanation is anisotropy in the scattering rate.

Assuming the latter, some interesting constraints are established for the scattering processes affecting in the Hall coefficient as follows,

\begin{enumerate}
    \item Preferentially affects the electron or the hole portions of the Fermi surface
    \item If it primarily affects the hole portions (along the $k=(\pi, \pi)$ vector), then the temperature dependence of the scattering weakens with doping
    \item If it primarily affects the electron portions (along the $k=(\pi, 0)$ vector), then the temperature dependence of the scattering strengthens with doping
    \item The scattering process is dominant below the temperature at which $R_H(\textrm{max})$ occurs, hardly affecting the data above this temperature
    \item The scattering process appears linear in temperature
\end{enumerate}

A natural continuation of this work would include a more precise determination of the low field region to determine with more certainty if the low temperature behaviour is truly $T$-linear as this would support the notion of an anisotropic parameter. A second possible experiment would be to measure the low temperature Hall in a material where the van-Hove does not have a significant momentum dependence in order to determine if the low temperature $R_H$ continued to decay down to a level less than the room temperature term.
