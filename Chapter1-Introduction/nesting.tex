

\section{Fermi surface nesting as a pairing mechanism}

The charge carrier in a superconducting condensate is a Cooper pair - a quasi-particle comprising of a bound state of two electrons with opposite spin and momentum. Given that due to their like charges two free electrons repel, it is natural to ask what could overcome the electromagnetic force to cause these electrons to remain bound in this state.

Bardeen, Cooper and Schreiffer in a 1957 paper\cite{Bardeen1957} detailed a pairing mechanism based around the concept of phonons scattering off ions which well suited the superconducting materials known at the time. It was not until the discovery of superfluidity in He$_3$ in 1972 ...

TODO: Cite He3 superfluidity discovery Oscheroff, Richardson. Leggett 1972

... that it became apparent that there may exist forms of pairing that were not phonon mediated. Later, superconducting analogues were found in the form of heavy Fermion materials and in 1986 when Bednorz and M\"uller\cite{Bednorz} discovered high transition temperature (\Tc) superconductivity in the cuprates. The cuprate class of materials that Bednorz and M\"uller found to be superconducting have \Tc~s far in excess of any previously known superconducting materials and although the Bardeen Cooper Schreiffer (BCS) phonon pairing may play a part, the predominant pairing mechanism in the \highTc materials is likely to be something else entirely\cite{Mazin2008}. There are several proposed mechanisms presently on offer including ...

TODO: What other pairing mechanisms are there, negative U?, CDW?

Of these theories, the one with arguably the most traction at present is that of spin-fluctuation mediated pairing. 

TODO: What actually is the cause of the attraction in the nesting picture? ... Spin fluctuation intereaction in real space is approximately propoprtional to the dipole interaction $V=-\mu . \mu \chi(r)$\cite{Bergemann2003} 

\subsection{Susceptibility}
    \label{Sec:1:NestingSusceptibility}

A well used measure of the nesting condition is the Lindhard susceptibility function. This is often quoted as,
\begin{equation}
\Chi_0(\vec{q}, \omega) = \lim_{\eta \to 0} \Sum_{\vec{k}}\Sum_{l,l\prime}\frac{f(\epsilon_{\vec{k}+\vec{q},l\prime}) - f(\epsilon_{\vec{k},l})}{\epsilon_{\vec{k}+\vec{q},l\prime} - \epsilon{\vec{k},l} - \hbar\omega - i\eta}|\langle \vec{k}+\vec{q},l\prime \mid  V \mid \vec{k},l \rangle|^2
\end{equation}
The numerator term contains two Fermi functions ($f(\epsilon) = 1/(\exp{\frac{\epsilon - \epsilon_F}{k_B T}} + 1)$) where $\epsilon$ and $\epsilon_F$ are the state energy and the Fermi energy respectively and $k_BT$ is the usual Boltzman energy conversion factor. These Fermi functions ensure that the susceptibilty is finite for states which scatter across the Fermi energy and zero if they do not. They also smear the calculations as a function of temeprature. The final term in the denominator is an artefact of the adiabatic approximation used to calculate the perturbation. The completed approximation takes the limit of $\eta \to 0$ which results in an expression for the imaginary part of Lindhard susceptibility, $\mathcal{Im}(\Chi_0) \propto \delta(\epsilon_{\vec{k}+\vec{q},l\prime} - \epsilon{\vec{k},l} - \hbar\omega)$ which, in a continuous calculation, results in resonances at excitations which match the difference in energies between states. However, in this thesis, the energy dispersions used to determine nesting conditions are not continuous and instead are based on discrete energies obtained from DFT calculations. As such $\eta$ will have to remain finite in order to broaden the delta function into a Lorentzian with width comparable to the energy differences between the discrete points -- the net result of this will be loss of some fine structure. The third term in the denominator corresponds to the excitation energy of the perturbing field with $\omega$ corresponding to the temporal frequency of the field. The first sum in the Lindhard function is over all $\vec{k}$ states in the first Brillouin zone. The DFT calculations do not provide values for all $\vec{k}$ states, instead a fairly coarse mesh evenly distributed over the Brillouin zone is used. The second sum combines each energy band. In practice only bands that lie close (within the adiabatic or temeprature broadening) to the Fermi energy need to be included in the calculations.

Peaks in this function correspond to scattering of states which cross the Fermi energy yet remain close to the Fermi energy.  We can derive this function by modelling an oscillatory perturbing field on a system. To solve to get an expression for the second order perturbation, we make the adiabatic limit approximation (i.e. the perturbing potential is gradually increase from zero at $t=\Infty$ to $v$ at $t=0$).

Although knowledge of the susceptibility is useful to model, for example, neutron scattering measurements, for our purposes we will use it to demonstrate the strength of particular nesting vectors in our example materials. For this reason we make the assumption that the transition matrix elements are unity, thus simplifying the equation calculations.


TODO: How does susceptibility tie in with nesting?
TODO: Lindhard susceptibility is a time dependent perturbation in the adiabatic limit to what? Adiabatic limit is where the pertubation time-frame is slow c.f. the unperturbed time-frame 
TODO: imaginary factor corresponds to the decay rate of the state
TODO: energy(susceptibility) is broadened by the decay rate




% The Stoner condition of $\mathcal{N}_0 I > 1$ -- where $\mathcal{N}_0$ is the density of states at the Fermi energy and $I$ is the molecular field constant, that scales the magnetism given a field -- indicates an energy instability\cite{Kubler2000}
