
\section{Fermi surface nesting as a pairing mechanism}

The charge carrier in a superconducting condensate is a Cooper pair - a quasi-particle comprising of a bound state of two electrons with opposite spin and momentum. Given that due to their like charges two free electrons repel, it is natural to ask what could overcome the electromagnetic force to cause these electrons to remain bound in this state.

Bardeen, Cooper and Schreiffer in a 1957 paper\cite{Bardeen1957} detailed a pairing mechanism based around the concept of phonons scattering off ions which well suited the superconducting materials known at the time. It was not until the discovery of superfluidity in He$_3$ in 1972 ...

TODO: Cite He3 superfluidity discovery Oscheroff, Richardson. Leggett 1972

... that it became apparent that there may exist forms of pairing that were not phonon mediated. Later, superconducting analogues were found in the form of heavy Fermion materials and in 1986 when Bednorz and M\"uller\cite{Bednorz} discovered high transition temperature (\Tc) superconductivity in the cuprates. The cuprate class of materials that Bednorz and M\"uller found to be superconducting have \Tc~s far in excess of any previously known superconducting materials and although the Bardeen Cooper Schreiffer (BCS) phonon pairing may play a part, the predominant pairing mechanism in the \highTc materials is likely to be something else entirely\cite{Mazin2008}. There are several proposed mechanisms presently on offer including ...

TODO: What other pairing mechanisms are there, negative U?, CDW?

Of these theories, the one with arguably the most traction at present is that of spin-fluctuation mediated pairing. 

TODO: What actually is the cause of the attraction in the nesting picture? ... Spin fluctuation intereaction in real space is approximately propoprtional to the dipole interaction $V=-\mu . \mu \Chi(r)$\cite{Bergemann2003} 

\subsection{Susceptibility}
    \label{Sec:1:NestingSusceptibility}

TODO: How does susceptibility tie in with nesting?




The Stoner condition of $\mathcal{N}_0 I > 1$ -- where $\mathcal{N}_0$ is the density of states at the Fermi energy and $I$ is the molecular field constant, that scales the magnetism given a field -- indicates an energy instability\cite{Kubler2000}
