
\section{Fermi surface nesting as a pairing mechanism}

The charge carrier in a superconducting condensate is a Cooper pair - a quasi-particle comprising of a bound state of two electrons with opposite spin and momentum. Given that due to their like charges two free electrons repel, it is natural to ask what could overcome the electromagnetic force to cause these electrons to remain bound in this state.

Bardeen, Cooper and Schreiffer in a 1957 paper\cite{Bardeen1957} detailed a pairing mechanism based around the concept of phonons within the structural lattice which well suited the superconducting materials known at the time. It was not until 1986 when Bednorz and M\:uller\cite{Bednorz} discovered superconductivity in a new class of materials that it became apparent that phonon mediated coupling was not the only pairing mechanism. The cuprate class of materials that Bednorz and M\:uller found to be superconducting have transition temperatures ($T_c$) far in excess of any previously known superconducting materials ...

TODO: What makes us believe that highTC is not phonon mediated?

... and although the Bardeen Cooper Schreiffer (BCS) phonon pairing may play a part, the predominant pairing mechanism in the \highTc materials is likely to be something else entirely. There are several proposed mechanisms presently on offer including ...

TODO: What other pairing mechanisms are there, negative U?, CDW?

Of these theories, the one with arguably the most traction at present is that of spin-fluctuation mediated pairing. 

TODO: What actually is the cause of the attraction in the nesting picture?

