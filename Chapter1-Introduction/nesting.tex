

\section{Fermi surface nesting as a pairing mechanism}

The charge carrier in a superconducting condensate is a Cooper pair - a quasi-particle comprising of a bound state of two electrons or two holes with opposite spin and momentum. Evidence for this configuration arises as a natural result of the Ginzberg-Landau model which, when applied to a superconducting system, gives the charge of the quasi-particle carriers as $2e$, where $e$ is the charge of an electron. Given that due to their like charges two free electrons repel, it is natural to ask what could overcome the electromagnetic force to cause these electrons to remain bound in this quasi-particle state.

Bardeen, Cooper and Schreiffer established much of the theoretical basis --- from which the Ginzberg--Landau model can be derived --- in \textit{BCS theory} (named after the authors) and within the framework of BCS theory, wrote a 1957 paper\cite{Bardeen1957} detailed a pairing mechanism known as the \textit{BCS model} which would explain how these electron remained bound together. The model is based around the concept of phonons scattering off ions which well suited the superconducting materials known at the time. Phenomenologically, the mechanism of attraction is reasonably straightforward. Electrons moving through a crystal lattice attract ions on the lattice sites. These heavy ions respond slowly and are drawn in \textit{behind} the electron. This has the effect of both screening the negative electron charge as well as providing an attractive positive potential for any electron following the original electron. The net effect is the leading electron draws the following electron along the same path, thus coupling them with one another. The wavelike distortion of the ions in the lattice can be considered as a phonon, and the interaction between the electrons and the lattice can be modelled as electron--phonon--electron scattering.

The BCS model on top of BCS theory accurately describes what we now know as \textit{conventional superconductivity}, that is pairing which forms a spin-singlet state ($S=0$) and which has zero orbital angular momentum ($L=0$). It was not until the discovery of superfluidity\footnote{Superfluidity and superconductivity share much of the same physics although rather than electrons or holes pairing, molecules pair instead. Parallels betwen the two are discussed in ref.\cite{Annett2010}} in $^3$He in 1972\cite{Osheroff1972} that it became apparent that there may exist forms of pairing that resulted in spin-triplet pairing state ($S=1$) with $L>0$. This was later confirmed when superconducting analogues were found in the form of heavy Fermion materials. What really spurred the explosion in interest though was the 1986 discovery by Bednorz and M\"uller\cite{Bednorz} of high transition temperature (\Tc) superconductivity in the cuprates and, more recently, the `pnictides' by Kamihara et al.\cite{Kamihara2008}. The cuprate class of materials that Bednorz and M\"uller found to be superconducting have \Tc~s far in excess of any previously known superconducting materials and although the BCS model phonon pairing may play a part, the predominant pairing mechanism in the \highTc materials is likely to be something else entirely.

\subsection{The case against conventional superconductivity in \highTc}

There is a great deal of evidence in the literature for non-BCS model pairing in the \highTc and heavy Fermion materials. Although the pairing wavefunction cannot be measured directly with current techniques, experiments indirectly infer \textit{unconventional} i.e. non s-wave, BCS-model, characteristics. For example, analysis on penetration depth measurements of YBa$_2$Cu$_3$O$_{7-\delta}$ show power law behaviour\cite{Annett1991}, indicating that there exists states within the momentum averaged gap. Moreover, SQUID measurements and Josephson tunneling experiments on the same material have confirmed alternating phase of the condensate wavefunction which points strongly to \DxTwoyTwo--wave symmetry\cite{VanHarlingen1994} (see also refs. therein). As for other cuprate materials, specific heat measurements on \BSCO\cite{Wang2011}, as well as peentration depth measurements on LSCO\cite{Froehlich1996} have also proved consistant with $d$-wave pairing. Further arguments have also been presented in the pnictide case based on calculations showing that the magnitude of the phonon pairing strength is not adequate for the high \Tc values attained\cite{Mazin2008}.

More evidence against conventional superconductivity include the unusual normal state (i.e. non-superconducting) state properties of all known \highTc and heavy Fermion materials. The BCS model is grounded in Landau Fermi liquid theory which models interacting itinerent electrons with quasiparticles of heavier effective mass than ordinary electrons and holes. A hallmark of Fermi liquid behaviour is a $B^2$ dependence of the magnetoresistance, however experiments on the cuprate La$_{2-x}$Sr$_{x}$CuO$_4$\cite{Cooper2009} and a heavy Fermion material\cite{Custers2003} have demonstrated fractional power law behaviour, $B^\gamma$ where $1 < \gamma < 2$, at temepratures above the superconducting tansition. Given that the Fermi liquid model breaks down in these examples, it follows that the BCS-model also is likely on shaky ground for these materials.

\subsection{Spin-fluctuations}

One possible 

\subsection{Pnictides}

Pnictide materials too show have a range of gap-symmetries with the most promising candidate being the s$_\pm$ state which   pentration depth measurements in LaFePO indicate that there are nodes in the gap function\cite{Fletcher2009},  spin is Measurements of nodes in the  of the  by  as to this include the fact that the cuprates and heavy Fermion materials both have normals state proprties which are 

Aside from the evidence of nodes in the superconducting gap in materials 

Some arguments against the BCS theory of pairing \cite{Haule2008,Yndurain2009,Mazin2008} based on arguments of 


FS nesting not the only cause of spin-fluctuations, also can be caused by frustrated superexchange for example TODO

Spin fluctuations mediate a repulsive interaction between Cooper pair candidates.

The anisotropic BCS equations specify that repulsive coupling between carriers can be pairing provided the order parameter changes sign over the coupling vector.


There are several proposed mechanisms presently on offer including charge fluctuations resulting in large ion polarisation \cite{Berciu2009}, however this was contested by Mazin and Schmalian\cite{Mazin2009}.


Of these theories, the one with arguably the most traction at present is that of spin-fluctuation mediated pairing. 

TODO: What actually is the cause of the attraction in the nesting picture? ... Spin fluctuation intereaction in real space is approximately propoprtional to the dipole interaction $V=-\mu . \mu \chi(r)$\cite{Bergemann2003} 

\subsection{Susceptibility}
    \label{Sec:1:NestingSusceptibility}

A well used measure of the nesting condition is the Lindhard susceptibility function. This is often quoted as,
\begin{equation}
\chi_0(\vec{q}, \omega) = \lim_{\eta \to 0} \sum_{\vec{k}}\sum_{l,l\prime}\frac{f(\epsilon_{\vec{k}+\vec{q},l\prime}) - f(\epsilon_{\vec{k},l})}{\epsilon_{\vec{k}+\vec{q},l\prime} - \epsilon{\vec{k},l} - \hbar\omega - i\eta}|\langle \vec{k}+\vec{q},l\prime \mid  V \mid \vec{k},l \rangle|^2
\end{equation}
respectively. The numerator term contains two Fermi functions ($f(\epsilon) = 1/(\exp{\frac{\epsilon - \epsilon_F}{k_B T}} + 1)$) where $\epsilon$ and $\epsilon_F$ are the state energy and the Fermi energy respectively and $k_BT$ is the usual Boltzman energy conversion factor. These Fermi functions ensure that the susceptibilty is finite for states which scatter across the Fermi energy and zero if they do not. They also smear the calculations as a function of temperature. The final term in the denominator is an artefact of the adiabatic approximation used to calculate the perturbation. The completed approximation takes the limit of $\eta \to 0$ which results in an expression for the imaginary part of Lindhard susceptibility, $\mathcal{Im}(\chi_0) \propto \delta(\epsilon_{\vec{k}+\vec{q},l\prime} - \epsilon{\vec{k},l} - \hbar\omega)$ which, in a continuous calculation, results in resonances at excitations which match the difference in energies between states. However, in this thesis, the energy dispersions used to determine nesting conditions are not continuous and instead are based on discrete energies obtained from DFT calculations. As such $\eta$ will have to remain finite in order to broaden the delta function into a Lorentzian with width comparable to the energy differences between the discrete points -- the net result of this will be loss of some fine structure. The third term in the denominator corresponds to the excitation energy of the perturbing field with $\omega$ corresponding to the temporal frequency of the field. The first sum in the Lindhard function is over all $\vec{k}$ states in the first Brillouin zone. The DFT calculations do not provide values for all $\vec{k}$ states, instead a fairly coarse mesh evenly distributed over the Brillouin zone is used. The second sum combines each energy band. In practice only bands that lie close (within the adiabatic or temeprature broadening) to the Fermi energy need to be included in the calculations.

Peaks in this function correspond to scattering of states which cross the Fermi energy yet remain close to the Fermi energy.  We can derive this function by modelling an oscillatory perturbing field on a system. To solve to get an expression for the second order perturbation, we make the adiabatic limit approximation (i.e. the perturbing potential is gradually increase from zero at $t=\infty$ to $v$ at $t=0$).

The real and imaginary parts of equation \ref{Eqn:1:Lindhard} are,
\begin{align}
\chi_0(\vec{q}, \omega)\prime &= \lim_{\eta \to 0} \sum_{\vec{k}}\sum_{l, l\prime}\frac{(\epsilon_{\vec{k}+\vec{q},l\prime} - \epsilon{\vec{k},l} - \hbar\omega) f(\epsilon_{\vec{k}+\vec{q},l\prime}) - f(\epsilon_{\vec{k},l})}{(\epsilon_{\vec{k}+\vec{q},l\prime} - \epsilon{\vec{k},l} - \hbar\omega)^2 + \eta^2}|\langle \vec{k}+\vec{q},l\prime \mid  V \mid \vec{k},l \rangle|^2 \\
\chi_0(\vec{q}, \omega)\prime\prime &= \lim_{\eta \to 0} \sum_{\vec{k}}\sum_{l, l\prime}\frac{-\delta f(\epsilon_{\vec{k}+\vec{q},l\prime}) - f(\epsilon_{\vec{k},l})}{(\epsilon_{\vec{k}+\vec{q},l\prime} - \epsilon{\vec{k},l} - \hbar\omega)^2 + \eta^2}|\langle \vec{k}+\vec{q},l\prime \mid  V \mid \vec{k},l \rangle|^2 \\
\end{align}
respectively.

Although knowledge of the susceptibility is useful to model, for example, neutron scattering measurements, for our purposes we will use it to demonstrate the strength of particular nesting vectors in our example materials. For this reason we make the assumption that the transition matrix elements are unity, thus simplifying the equation calculations.


TODO: How does susceptibility tie in with nesting?
TODO: Lindhard susceptibility is a time dependent perturbation in the adiabatic limit to what? Adiabatic limit is where the pertubation time-frame is slow c.f. the unperturbed time-frame 
TODO: imaginary factor corresponds to the decay rate of the state
TODO: energy(susceptibility) is broadened by the decay rate




% The Stoner condition of $\mathcal{N}_0 I > 1$ -- where $\mathcal{N}_0$ is the density of states at the Fermi energy and $I$ is the molecular field constant, that scales the magnetism given a field -- indicates an energy instability\cite{Kubler2000}
