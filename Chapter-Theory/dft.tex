
\section{Density functional theory}
\label{Sec:Theo:Dft}

The interpretation of the \ac{dHvA} measurements presented later in this thesis rely to some extent on the ab-initio calculation of the energy bands of \BaFeP{} using the \WIEN{} code~\cite{Blaha2001} --- the technique used to find these energy dispersions are based on a \ac{DFT} scheme. The following is broad overview of \ac{DFT} which is drawn from notes from a series of summer school lectures by M. L\"uders~\cite{Luders2010} and the `ABC of DFT' by K. Burke~\cite{Burke2003}.

 Although implementations of \ac{DFT} rely on various approximations, \ac{DFT} itself has been shown to be exact and mathematically rigorous. It comprises of a set of theorems developed and proven by Hohenberg, Kohn and Levy~\cite{Hohenberg1964, Levy1979} and a methodology for solving to obtain the ground state energies developed by Kohn and Sham. The principle theorem outlined by \ac{HK} shows that the ground state external potential, $v_{\textrm{ext}}(\vect{r})$, of a system can be determined by the ground state density, $n(\vect{r})$, alone and vice-versa. A second \ac{HK} theorem outlines a minimisation condition which expresses the ground state energy as follows,
\begin{equation}
\label{Eqn:Theo:HKMinimisation}
\frac{\partial F[n]}{\partial n(\vect{r})} + v_{\textrm{ext}}(\vect{r}) = \mu,
\end{equation}
where,
\begin{equation}
F[n] = T[n] + V_{\textrm{ee}}[n]
\end{equation}
$F[n]$ is the `universal' functional\footnote{A functional maps a function onto a single vector or scalar --- typically by integrating over the function --- and is commonly denoted with the function parameter in square brackets. Compare this with the definition of a function which maps a series of scalars onto a single scalar or vector.} and $T[n]$ and $V_{\textrm{ee}}[n]$ are the kinetic and correlation functionals respectively. $\mu$ is the chemical potential which is introduced as a normalisation term that ensures that there are an appropriate number of electrons in the charge density. The universal functional is so called because the system is completely defined in the external potential term alone and so $F[n]$ is common to all systems, nonetheless it still requires approximation. For this reason as well as the fact that there are no clues from the \ac{HK} theorems as to a good starting form for $n$, still means the problem is intractable.

Kohn-Sham developed a method to find a good starting form for $n$ by showing that there exists a pseudo-potential, $v_{\textrm{KS}}$, that satisfies the above equation for a \emph{non-interacting} system, i.e. $F[n] = T[n]$, which shares the same $n$ as the original interacting system. This abstract potential, which takes the place of $v_{\textrm{ext}}$ in the above equation, has no strict physical meaning but it allows us to build a common expression for $n$ in terms of a sum of single particle wavefunctions. It is given as follows,
\begin{equation}
\label{Eqn:Theo:KohnShamPotential}
v_{\textrm{KS}} = v_{\textrm{ext}}(\vect{r}) + \int d^3r^\prime\frac{n(\vect{r})}{|\vect{r}-\vect{r}^{\hspace{2px}\prime}|} + \frac{\partial E_{\textrm{xc}}}{\partial n(\vect{r})}
\end{equation}
where $E_{\textrm{xc}}$ is the combined particle correlation and exchange energy terms which is approximated according to the type of problem to be solved\footnote{Note that in principle, these two terms should be separate but most approximations tend to combine them into a single term.}.  Once an approximate form for the exchange term has been chosen, the above can be used to find the ground state energy by running a self consistency cycle that forms the heart of most \ac{DFT} codes,
\Needspace*{5\baselineskip} % Ensure this appears on same page
\begin{enumerate}
    \item Guess an initial $n_{i=0}$
    \item Calculate $v_{\textrm{KS}}$ from $n_{i}$ using eqn.~\ref{Eqn:Theo:KohnShamPotential}
    \item Minimise the non-interacting (Kohn-Sham) form of eqn.~\ref{Eqn:Theo:HKMinimisation}
    \item Calculate the new $n_{i+1}$
    \item Repeat from step 2 with $n_i$ being adjusted by $n_{i+1}$ until $n_{i} - n_{i+1} < \textrm{some tolerance}$
\end{enumerate}
Simply replacing $n_{i}$ with $n_{i+1}$ can cause the calculation to rapidly diverge and so instead a mixing scheme is used. Typically this incorporates a small fraction of $n_{i+1}$ with the rest made up of the old $n_{i}$. A more complex mixing scheme can be employed to ensure more rapid convergence, the Broyden mixing scheme for example uses a Newton-Raphson style root finding mechanism on the Jacobian of the $n_{i} - n_{i+1}$~\cite{Broyden1965}.


\subsection{The generalised gradient approximation}

The correlation term represents the most significant approximation in the calculation of \ac{DFT}. For the \ac{DFT} presented in this thesis, the \ac{GGA} was used which is part of the family of \acp{LDA}. The simplest (i.e. lowest order) \ac{LDA} takes the effects of the electron-electron correlations at point $\vect{r}$ to be constant throughout the system with the magnitude based on the charge density at $\vect{r}$. This works particularly well for free-electron-like systems with lots of itinerant valence electrons since the electrons are evenly spread throughout --- however it does not work so well for highly localised Hubbard-like systems where there is a high density of electrons at atomic sites, but very little density just off the sites. A step towards improving this comes by raising the order of the approximation so that it modifies the constant local density with the rate of change of the local density as you move off the site (i.e. the local charge density gradient). It turns out however that incorporating the gradient results in less accuracy than the simple \ac{LDA} due to the \ac{LDA} `accidentally' cancelling a series of so called sum rules. The \ac{GGA} builds on the higher order gradient approximation by incorporating the cancelling of the sum rules to obtain a reasonably accurate approximation to the correlation potential. 

The precise way to express the \ac{GGA} however is still a matter of debate though with there being multiple implementations~\cite{Perdew1996, Perdew1986}, each of which may give slightly different results\footnote{See for example table 1 in ref.~\cite{Perdew1996}}. Nonetheless \acp{GGA} tends to perform better than zeroth order \ac{LDA} with inhomogeneous electron densities.

\subsection{Single particle wavefunction bases}

Typically, close to the atom, electrons tends to have a radial symmetry whereas itinerant valence electrons are more planewave-like. Matching each of these to an appropriate single particle basis set dramatically reduces the amount of calculation time. The \ac{APW} method defines a series of `muffin-tin spheres' which are centred on each of our atoms. Those well inside are described in terms of radial basis orbitals, those well outside are described in terms of plane waves.

Andersen further simplified the radial basis portion of the \ac{APW} method by approximating the wavefunctions by a first order Taylor expansion with respect to energy~\cite{Andersen1975}. The result is known as \ac{LAPW}.

\subsection{Local orbit wavefunction bases}

The standard \ac{LAPW} method can be made more efficient by up to an order of magnitude if additional wavefunctions are included with the standard \ac{APW} wavefunctions which better describe the states close to the edge of the muffin tin spheres known as the `semi-core states'.

These additional wavefunctions are known as `local orbitals' and allow the calculation to relax the condition that the standard \ac{APW} wavefunctions must have a continuous slope at the muffin tin boundary. The local orbit wavefunctions are used to smooth over the kinks where the plane wavefunctions and the radial wavefunctions meet. The functions are radial in nature but are $\vect{k}$ independent. Including these wavefunctions can result in up to \unit{50}{\%} fewer wavefunctions required for convergence and significantly shorter calculation times~\cite{Madsen2001}.

\subsection{Code and execution details}

Calculations presented in this thesis were performed using \WIEN{} version 07.2 (20th Feb 2007)~\cite{Blaha2001} using \ac{LAPW} without the local orbitals. Unless specified, non-spin orbit calculations are presented although spin-orbit calculations were checked and did not show significant differences. The \ac{GGA} according to Perdew-Burke-Ernzerhof~\cite{Perdew1996} was used for the exchange correlation functional.

Preprocessing of the \WIEN{} data into voxel form as well as the theoretical angle plots were performed using a modified version of MATLAB code written by Dr. E. Yelland. The basis for the code has been thoroughly field tested within the group over a number of years.

