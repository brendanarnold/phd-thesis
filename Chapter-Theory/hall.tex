
\section{Hall effect}

The Hall effect is a simple consequence of the Lorentz force on a moving charge. As an electron (hole) moves along a rectangular slab subject to a perpendicular magnetic field, the electrons (holes) are deflected to one side of the slab. Eventually the charge density one one side becomes high enough that the Coulomb repulsion force of the density on subsequent charge carriers balances the Lorentz force and an equilibrium voltage between either side of the slab is reached. This voltage is known as the Hall voltage, $V_H$ and is given by,
\begin{equation}
    V_H = -\frac{IB_{\perp}}{ned}
\end{equation}
where $I$ and $B_{\perp}$ are the current and perpendicular magnetic field and $n$, $e$ and $d$ are the carrier density, charge and slab thickness respectively. $V_H$ is what is measured in our experiment. This is usually further abstracted to the Hall coefficient, $R_H$, which encapsulates the carrier density as follows,
\begin{equation}
    R_H = \frac{V_H d}{IB_\perp} = \frac{1}{ne}
\end{equation}

