

\section{Fermi liquid theory}

The nearly-free electron gas model for ordinary metals\footnote{A model which only considers the kinetic energies of the electrons and Pauli exclusion terms} is an extremely coarse approximation to the real situation and yet provides surpringly good, predictive results in a variety of scenarios. Fermi liquid theory provides the theoretical basis which explains why we can use non-interacting particle models with a simple modification of the masses of the interacting Fermionic particles.

From a matchematical standpoint, Fermi liquid theory considers a gas of non-interacting particles and gradually `switches on' the interactions. Provided the system transitions adiabatically\footnote{i.e. with no symmetry breaking changes in phase or, in other words, there is a one-to-one mapping of the particles in the initial non-interacting system to the quasiparticles in the final interacting system} then the `particles' in the resulting system, which is known as a \emph{Fermi liquid}, can be modeled using the same mathematics as the non-interacting system with an adjusted mass. This adjusted mass is known as the \emph{enhanced mass} and encompasses the interactions in the system. As a rule, the larger the enhancement the greater the interaction strength. The enhanced mass particles are labeled quasiparticles since they no longer share the same mass as an electron at rest and are, arguably, a product of a mathematical abstraction.

At the time of writing, Fermi liquid theory describes what would be considered `ordinary' metals with deviations from Fermi liquid theory generally considered of interest in a number of systems. Moreover the theory behind measurement techniques such as \ac{dHvA} --- decribed in the next section --- rely on the existence of coherent quasiparticles at the Fermi surface to be valid. The reconscilliation of observed \ac{dHvA} oscillations in cuprates with evidence for reduced quasiparticle weight from \ac{ARPES} data currently provides one of the interesting challenges of \highTc research.
