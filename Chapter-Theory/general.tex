
\section{Band theory}
    \label{Sec:Theo:BandTheory}

The size and shape of the Fermi surface is key to many electronic phenomena and in particular plays an important role in the formation of the \acf{SDW} instability explained in section~\ref{Sec:Theo:SpinDensityWave}. The Fermi surface, defined as the volume in $k$-space which bounds the occupied electron states at $T=\unit{0}{\kelvin}$, is spherical in simple free-electron case. This corresponds with intuition as the electrons tend to the lowest energy states without double occupancy and in the free-electron case the lowest states correspond to the lowest magnitude of $\vect{k}$. However in a lattice, this is rarely the case and instead  

\section{Fermi liquid theory}
    \label{Sec:Theo:FermiLiquidTheory}

The nearly-free electron gas model for ordinary metals\footnote{A model which only considers the kinetic energies of the electrons and Pauli exclusion terms.} used as the foundation for the \ac{dHvA} theory in the next section is an extremely coarse approximation to the real situation and yet provides surprisingly good, predictive results in a variety of scenarios even though the electronic lattice potential is ignored. Fermi liquid theory provides the theoretical basis which explains why we can use non-interacting particle models with a simple modification of the masses of the interacting Fermionic particles.

From a mathematical standpoint, Fermi liquid theory considers a gas of non-interacting particles and gradually `switches on' the interactions. Provided the system transitions adiabatically\footnote{Adiabatic in this context means with no symmetry breaking changes in phase or, in other words, there is a one-to-one mapping of the particles in the initial non-interacting system to the quasiparticles in the final interacting system.} then the `particles' in the resulting system, which is known as a `Fermi liquid', can be modelled using the same mathematics as the non-interacting system with an adjusted mass. This adjusted mass is known as the `enhanced mass' and encompasses the interactions in the system with the magnitude being an indicator of the interaction strength. The enhanced mass particles are labelled quasiparticles since they no longer share the same mass as an electron at rest and are, arguably, a product of a mathematical abstraction.

At the time of writing, Fermi liquid theory describes what would be considered `ordinary' metals with deviations from Fermi liquid theory generally considered of interest in a number of systems. Moreover the theory behind measurement techniques such as \ac{dHvA} --- described in the next section --- rely on the existence of coherent quasiparticles at the Fermi surface to be valid. The reconciliation of observed \ac{dHvA} oscillations in cuprates with evidence for reduced quasiparticle weight from \ac{ARPES} data currently provides one of the interesting challenges of high-$T_c$ research.

