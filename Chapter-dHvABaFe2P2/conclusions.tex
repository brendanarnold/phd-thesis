
\section{Conclusions}

The \BaFeP crystal and the subsequent angle dependent \ac{dHvA} measurements are of very good quality as evidenced by a number of traits including the presence of second and third harmonics in the \acp{FFT}, the hole orbits showing up over a wide angular range, the early onset of oscillations at \unit{6}{\tesla} and the observation of the Zeeman splitting of the \ac{FFT} peaks. The crystal appears to be a very clean single crystal although there is some evidence of some misaligned domains, for example from some of the Bragg spots doubling up in the \ac{XRD} and multiple peaks observed in the \ac{FFT} at particular angles -- see for example $\alpha$ towards the $[100]$ direction above $\sim 20\degree$ in figure~\ref{Fig:ResD:AngleSweepMeasured}. There are approximately half a dozen separate peaks observed at this location which implies a similar amount of misaligned domain orientations. This misalignment however does not appear to affect the overall data which largely does not resolve the extra domains.

The Fermiology is largely solved by the angle plots with only a few minor ambiguities.  The $F\cos \theta$ angle dependent plots clearly show approximately level curves for the two hole Fermi surfaces demonstrating that $\alpha$ and $\beta$ are approximately two-dimensional. The hole surface, $\gamma$, deviates at high angles and $\delta$ is strongly three dimensional. Although we cannot say for certain whether $\gamma$ is pinched off or not, based on the rigidly shifted \ac{DFT} we expect that the minima to be small but not zero and was not observed due to low frequency noise in the oscillations.

Like previous measurements of band structure by \ac{dHvA}, the Fermi surfaces are smaller than predicted by \ac{DFT} calculations\cite{Shishido2010, Analytis2010c}. Ortenzi \etal\cite{Ortenzi2009} posits an explanation based on 


Initial comparisons with \ac{dHvA} measurements on BaFe$_2$(As$_{0.37}$P$_{0.63}$)$_2$ by Analytis \etal shown in figure~\ref{Fig:ResD:IdentifyingBands} lead us to believe that the smaller hole orbit in the paper was misidentified as the smaller and in fact represents


Although band 2 begins to become more \DzTwo-like as it approaches the bulge\footnote{Refer back to the tenth panel in figure~\ref{Fig:ResD:Band2DCharacterVsKz} for the orbital character of band 2.}, the nesting vectors occur at...
Both surfaces have different \kz dispersions which mean nesting along the vector of $q=(\pi, \pi, \pi/2)$ is only partial thus providing the right conditions necessary for possible pair forming spin-density-wave fluctuations. \TODO{Does non-z orbital character preclude nesting with a z component?}


Similar differences between the measured data and calculations were observed for the sister 122 compound SrFe$_2$P$_2$\cite{Analytis2009} and is entirely consistent with results obtained on the entire \BaFePAs series\cite{Shishido2010}. Notably however, the non-nested 122 pnictide compound CaFe$_2$P$_2$ does not show any such differences to the DFT calculations, suggesting that perhaps the shifts in energy may arise from spin-fluctuations.

Orbital character is negligible for s and p orbitals for all bands which confirms ...

Energy shifts for band $2$ are proprtional to the \DzTwo and \DxzDyz characters, implies there may be a link between the \kz scattering and energy enhancements.


It is interesting to note that the DFT applied energy shifts apply to partially nested Fermi surfaces, whereas the large, unested portion of band 2 has zero shift. Other partially nested pnictide materials such as LaFePO\cite{Carrington2009} and SrFe$_{2}$P$_{2}$\cite{Analytis2009} have similar shifts whereas the non-nested, material CaFe$_{2}$P$_{2}$\cite{Coldea2009} matches the DFT calculations well without any rigid energy shift. This correlation between nesting and corrections to the DFT calculation lends weight to the notion that the shifts in the calculation energies are cause by spin fluctuations.


