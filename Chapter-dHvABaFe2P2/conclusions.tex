
\section{Discussion}

The \BaFeP crystal and the subsequent angle dependent \ac{dHvA} measurements are of very good quality as evidenced by a number of traits including the presence of second and third harmonics in the \acp{FFT}, the hole orbits showing up over a wide angular range, the early onset of oscillations at \unit{6}{\tesla} and the observation of the Zeeman splitting of the \ac{FFT} peaks. The crystal appears to be a very clean single crystal although there is some evidence of some misaligned domains, for example from some of the Bragg spots doubling up in the \ac{XRD} and multiple peaks observed in the \ac{FFT} at particular angles -- see for example $\alpha$ towards the $[100]$ direction above $\sim 20\degree$ in figure~\ref{Fig:ResD:AngleSweepMeasured}. There are approximately half a dozen separate peaks observed at this location which implies a similar amount of misaligned domain orientations. This misalignment however does not appear to affect the overall data which largely does not resolve the extra domains.

The Fermiology is largely solved by the angle plots with only a few minor ambiguities.  The $F\cos \theta$ angle dependent plots clearly show approximately level curves for the two hole Fermi surfaces demonstrating that $\alpha$ and $\beta$ are approximately two-dimensional. The hole surface, $\gamma$, deviates at high angles and $\delta$ is strongly three dimensional. Although we cannot say for certain whether $\gamma$ is pinched off or not, based on the rigidly shifted \ac{DFT} we expect that the minima to be small but not zero and was not observed due to low frequency noise in the oscillations.

Previous \ac{dHvA} measurements on BaFe$_2$(As$_{0.37}$P$_{0.63}$)$_2$ by Analytis \etal shown in figure~\ref{Fig:ResD:IdentifyingBands} identified the branch of the \ac{dHvA} angle data at around \unit{500}{\tesla} as the neck of the 2D hole pocket. In our own analysis, it made much more sense to attribute this curve to the neck portion of the 3D hole band, with the neck of the 2D pocket being buried in the low frequency noise. These two different statements are not necessarily incompatible. Since the \ac{DFT} data for the entire series suggests that whilst the 2D hole pocket retains the same shape along the \BaFePAs series, the 3D hole pocket narrows considerably at the $\Gamma$ point and switches from being concentric with the 2D pocket at $x=1$ to crossing through the 2D pocket as $x$ is reduced. However when the spin-orbit interaction is considered in the \ac{DFT} calculations, the crossing of the surfaces causes the bands to be redifined such that the bands do not actually cross, in which case there would not be a Yamaji point as specified in the Analytis paper since the similarly sized orbits at \unit{50}{\degree} angle are not from the same band. An attempt was made to see if there was an enhacement of the oscillation amplitude at the correpsonding putative Yamaji point in the \BaFeP data but the close proximity of the strong oscillations from the electron pockets made this intractable.

\TODO{The nesting}
% Although band 2 begins to become more \DzTwo-like as it approaches the bulge\footnote{Refer back to the tenth panel in figure~\ref{Fig:ResD:Band2DCharacterVsKz} for the orbital character of band 2.}, the nesting vectors occur at...

% Both surfaces have different $k_z$ dispersions which mean nesting along the vector of $q=(\pi, \pi, \pi/2)$ is only partial thus providing the right conditions necessary for possible pair forming spin-density-wave fluctuations. \TODO{Does non-z orbital character preclude nesting with a z component?}
% Orbital character is negligible for s and p orbitals for all bands which confirms ...

% Energy shifts for band $2$ are proprtional to the \DzTwo and \DxzDyz characters, implies there may be a link between the $k_z$ scattering and energy enhancements.

Like previous measurements of band structure by \ac{dHvA} in the \BaFePAs materials, the Fermi surfaces are smaller than predicted by \ac{DFT} calculations\cite{Shishido2010, Analytis2010c}. Ortenzi \etal\cite{Ortenzi2009} posits an explanation based on interband scattering which leads to shrinking of the electron and hole pockets and an enhancement of the effective mass based on relaxing an assumption on the chemical potential being far\footnote{i.e. greater than the scattering boson energy scale} from the electron/hole band edges. Similar moderate effective mass enhacements to what we found in \BaFeP of around $1.4m_e$ were calculated --- albeit modeled on a more two dimensional pnictide, LaFePO --- along with the fact that the theory predicts stronger shifts where interband coupling occurs is supported by the \BaFeP data. The nested portions of the 3D $\delta$ hole band, for example, is strongly shifted where it nests with the electron band but the bulge which does not nest with anything is nto shifted at all. Similar shifts between the measured data and calculations were observed for the sister 122 compound SrFe$_2$P$_2$\cite{Analytis2009} which is also a partially nested material and yet shifts are notably absent for the non-nested 122 pnictide compound CaFe$_2$P$_2$~\cite{Coldea2009} which matched the \ac{DFT} calculations with no ajdustments to the energies. Ortenzi's paper is not specific as to the kind of scattering other than it is mediated via a bosonic mode.

It is not clear at this stage whether the shifting of the Fermi surface proprtional to the electron character for the 3D hole $\delta$ band performed in section~\ref{Sec:ResD:ShiftingDFTPropToOrbitalCharacter} represents anything physical or is simply a convenient and reproducable way to obtain the correct band topology. Settling this question will require further investigation. 

Similarly the harmonic fits represented in section~\ref{Sec:ResD:TightBindingFits} for the 3D hole pocket at least may simply represent a convenient way to obtain the Fermi surface topology.


