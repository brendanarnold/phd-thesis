\documentclass[11pt, a4paper]{article}

\usepackage{units}


\title{The Fermiology of BaFe$_2$P$_2$ and Charge Transport in Widely Doped BSCO$2201$}
%Bi$_{2+z-y}$Pb$_{y}$Sr$_{2-x-z}$La$_{x}$CuO$_{6+\delta}$}
\author{Brendan. J. Arnold}

\begin{document}

% Include the shorthand for common complex phrases
% Meta macros
\definecolor{Highlight}{RGB}{255,255,128}
\sethlcolor{Highlight}
\newcommand{\TODO}[1]{\hl{TODO: (#1)}}
\newcommand{\unitspacing}{\;}
\newcommand{\code}[1]{\lstinline!#1!} % n.b. this will fail if code snippet has delimiter character in it
\newcommand{\nt}[1]{\emph{#1}}
% Materials
\newcommand{\BaFeP}{BaFe$_{2}$\-P$_{2}$}
\newcommand{\SrFeP}{SrFe$_{2}$\-P$_{2}$}
\newcommand{\BaFeAs}{BaFe$_{2}$\-As$_{2}$}
\newcommand{\BaFeAsP}{BaFe$_2$(As$_{1-x}$\-P$_{x}$)$_{2}$}
\newcommand{\BaCoFeAs}{BaCo$_{2x}$\-Fe$_{2(1-x)}$As$_{2}$}
\newcommand{\BaKFeAs}{Ba$_{x}$K$_{(1-x)}$\-Fe$_2$As$_{2}$}
% \newcommand{\BSCO}{\ac{BSCO}}
% Orbital characters
\newcommand{\totD}{$\Sigma d$}
\newcommand{\DzTwo}{\ensuremath{d_{z^2}}}
\newcommand{\Dxy}{\ensuremath{d_{xy}}}
\newcommand{\DxTwoyTwo}{\ensuremath{d_{x^2-y^2}}}
\newcommand{\DxzDyz}{\ensuremath{d_{xz}}+\ensuremath{d_{yz}}}
% General phrases
\newcommand{\WIEN}{WIEN2k}
\newcommand{\etal}{\textit{et al.}}
% Various notation
\newcommand{\Real}{\ensuremath{\operatorname{Re}}}
\newcommand{\Imag}{\ensuremath{\operatorname{Im}}}
\newcommand{\ten}[1]{\ensuremath{\times 10^{#1}}}
% Some contentious notation
% \newcommand{\vect}[1]{\vec{#1}}      % Arrows over vectors
\newcommand{\vect}[1]{\boldsymbol{#1}}   % Bold vectors
\newcommand{\op}[1]{\boldsymbol{\hat{#1}}}
% New environment for chapter introductory abstract
\newenvironment{chapterabstract}
{\rule{\linewidth}{0.15mm}\hspace{\stretch{1}} \\}
{\\ \hspace{\stretch{1}}\rule{\linewidth}{0.1mm}}

% Define new command for typesetting mathematical units in mathmode - see 
% http://termos.vemod.net/typesetting-units-in-latex
% \newcommand{\unit}[1]{\ensuremath{\, \mathrm{#1}}}


\maketitle


\section{Unresolved issues in High-Tc superconductivity}
    \subsection{Fermi surface nesting as a pairing mechanism}
        \begin{itemize}
            \item Describe nesting and link to susceptibility as an intro to BaFe2P2 dHvA results
        \end{itemize}
    \subsection{The pseudogap versus the coherent state}
        \begin{itemize}
            \item Describe current theories of `friend' or `foe' of superconductivity as intro to BSCO magnetoresistance results
        \end{itemize}
    \subsection{Doping determination}
        \begin{itemize}
            \item Discuss problem of doping determination, LSCO and comparisons with other cuprates as intro to BSCO Hall results
        \end{itemize}


\section{Experimental technique}

    \subsection{Measuring charge transport}
        \subsubsection{Fermi liquid theory}
            \begin{itemize}
                \item Description of basic Fermi liquid theory to provide a contrast to behaviour observed in high-Tc superconductors
            \end{itemize}
        \subsubsection{Hall effect}
            \begin{itemize}
                \item Briefly discuss theory of Hall effect
            \end{itemize}
        \subsubsection{Magnetoresistance}
            \begin{itemize}
                \item Briefly discuss theory of magnetoresistance
            \end{itemize}
        \subsubsection{Six probe technique}
            \begin{itemize}
                \item Discuss geometry
                \item Mention high field measurements
                \item Briefly discuss Polo setup, amplifier characterisation
                \item Show Cu hall bar results (to be performed)
            \end{itemize}
        \subsubsection{Sample size determination}
            \begin{itemize}
                \item Discuss briefly FIB and optical microscope
            \end{itemize}

    \subsection{dHvA torque oscillation}
        \subsubsection{Theory}
            \begin{itemize}
                \item Briefly describe Landau levels, Lifshitz-Kosevich equation, Onsager relation
                \item Some discussion as to limitations, i.e. non-superconducting state, only extremal areas, relatively long mean free path required etc.
            \end{itemize}
        \subsubsection{Method}
        \begin{itemize}
            \item Brief discussion of use of AFM cantilevers
            \item Discuss new sample mounting (grease)
            \item Introduce novel temperature correction techniques for use on Yellow Magnet
            \item Discuss angle determination using x-rays, coil and data symmetry
        \end{itemize}


\section{dHvA measurements on \BaFeP}

    \subsection{The \BaFePAs series}
        \begin{itemize}
            \item Discuss phase diagram, cleanliness of samples, pressure analogy
            \item Introduce existing measurements in Shishido, Analytis papers
            \item Present x-rays showing crystal quality
        \end{itemize}

    \subsection{Angle dependent measurements}
        \subsubsection{Mapping the Fermi surface with DFT calculations}
            \begin{itemize}
                \item Angle measurement results
                \item DFT calculations, shifting to fit
                \item Demonstration of the nesting 
            \end{itemize}
        \subsubsection{Susceptibility calculations}
            \begin{itemize}
                \item Explanation of Lindhard function
                \item Results of calculations in various conditions (i.e. $T=0\unit{K}$, $T=300\unit{K}$)
            \end{itemize}    

    \subsection{Measuring effective mass}
        \subsubsection{Analysis techniques}
            \begin{itemize}
                \item Discuss simple LK fits
                \item Discuss correcting for large field range using `retrofitting'
                \item Discuss correcting for large field range using `microfits'
            \end{itemize}
        \subsubsection{Effective mass results}

    \subsection{Determining the spin mass}
        \begin{itemize}
            \item Discuss mass enhancement at the Fermi level due to spin excitations
            \item Fits to the peak amplitude and possible values for $m_{s}$
        \end{itemize}
            

    \subsection{Conclusions}
        \begin{itemize}
            \item Discussion as to possible origins of shift (compare with results on CaFe2P2 and SrFe2P2)
            \item Discussion of nesting conditions in non-superconducting end member
            \item Discuss relatively small mass enhancements
            \item Compare with previous results by Analytis et al. i.e. lack of Yumaji point, misidentification of bands        
        \end{itemize}


\section{Hall measurements on \BSCO}

    \subsection{Field sweeps}
        \begin{itemize}
            \item Compare techniques for determining the linear portion of data
            \item Show comparison of BSCO with existing LSCO and Thallium data and determine doping
        \end{itemize}

    \subsection{Conclusisons}
        \begin{itemize}
            \item Discuss validity and utility of method
        \end{itemize}
    

\section{Magnetoresistance measurements on \BSCO}

    Recreate Pat's original analysis in a similar way to that in the LSCO paper and upcoming BSCO paper.

    \subsection{Temperature sweeps}
        \begin{itemize}
            \item Show divergence in the differential as indicator of the onset of the decay of the pseudogap
            \item Mention use of temperature sweeps for minor temperature corrections in the field sweeps
        \end{itemize}
    \subsection{Field sweeps}
        \begin{itemize}
            \item Discuss techniques for determining $H_2$, i.e. the $B^2$ portion of the magnetoresistance curve. Look to alternate techniques to one Pat used
            \item Present data with comparisons to high field results from Toulouse and Nijmegan performed by Pat, XF and Ionna
            \item Discuss multi-carrier model fitting similar to analysis performed on YBCO
            \item Results on low temperature ($<\unit[300]{K}$) resistivity
            \item High temperature BSCO if necessary -- results not yet taken
        \end{itemize}

    \subsection{Conclusions}
        \begin{itemize}
            \item Discuss the further refinements to the phase diagram of BSCO and what this means for the pseudogap
        \end{itemize}

    \section{Bibliography}

\end{document}

