\thispagestyle{empty}\null\clearpage

\setcounter{page}{1}
\pagenumbering{roman}


\section*{Abstract}


This thesis presents \ac{dHvA} measurements on high quality samples of \BaFeP. Energy dispersions from \ac{DFT} calculations were tweaked to match the measured Fermi surface orbits by rigidly shifting both the inner and outer electron energy bands and the inner hole band by \unit{0.0050}{\textrm{Ry}}, \unit{0.0043}{\textrm{Ry}} and \unit{-0.0083}{\textrm{Ry}} respectively. The hourglass shaped outer hole band energy required no shift at the wide part and a shift of \unit{-0.0038}{\textrm{Ry}} at the narrow part which was found to be nested with the inner electron band. To achieve a smooth transition between the two energy shift regimes, the dispersion was tweaked proportionally to the \DzTwo band character and a complete Fermi surface was determined. The shifts were attributed to nesting conditions which were supported from calculation of the bare Lindhard susceptibility. Therefore nesting is not a sufficient condition for superconductivity.

The thermal effective masses were determined on the electron and, for the first time, hole orbits and the spin effective masses on the electron orbits. The masses showed a moderate renormalisation (between $\unit{0.88}{m_b}$ and $\unit{3.04}{m_b}$) on both hole and electron bands in line with previous literature.

In addition, Hall measurements taken in high field from \unit{1.4}{\kelvin} to \unit{300}{\kelvin} on good quality samples of \ac{BSCO} were presented. A sharp change in $R_H(\unit{0}{\kelvin})/R_H(\unit{300}{\kelvin})$ was observed at $p=0.19$ well inside the superconducting dome which coincides with various phenomena related to the pseudogap and so hints at the fact that the pseudogap may persist inside the superconducting dome.

A simple model based on the Ong construction~\cite{Ong1991} was fitted to the Hall data and relative magnitudes of the scattering terms, $\Gamma = \Gamma_0 + \Gamma_1 \cos(2\phi)^2 T + \Gamma_2 T^2$ were compared with terms obtained from fits to resistivity curves. The $T$-linear terms were found to agree within a factor of around $0.6$ to $1.5$ although the residual resistivity terms only agreed within an order of magnitude likely due to a lack of a $v_F$ term in the scattering rate and the relative proximity to the van-Hove singularity. Nonetheless an increase in scaling of the $T$-linear term with $T_c$ similar to that found by Abdel-Jawed \etal~\cite{Abdel-Jawad2007}. These results provide a good starting point for further refinement and possible full agreement between temperature dependent Hall and resistivity data in the cuprates without resorting to complicated Fermi surface reconstruction scenarios.

Finally a novel doping determination technique based on a method of matching high temperature Hall coefficient of \ac{BSCO} to that of \ac{TL2201} as a reference material. The resulting dopings are greater than those from the `universal' Presland/Tallon relation~\cite{Presland1991} and less than those assigned to similar samples measured with \ac{ARPES}~\cite{Kondo2004}. The overall spread in dopings for these samples from this new method was determined to be between $p=0.12$ and $p=0.36$.

% 300 words
