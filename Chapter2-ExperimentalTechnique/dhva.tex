
\section{dHvA torque oscillation}

\subsection{Theory}

\begin{equation}
A(B,\phi,\theta,T)=A_0R_{\Gamma}R_TR_DR_{\textrm{mos}}R_{\textrm{dop}}R_sR_{\textrm{warp}}
\label{Eqn:2:OscilllationAmp}
\end{equation}

\begin{align}
R_T(B, \theta, T) &= \frac{X}/{\sinh{X}} \\
X                 &= \frac{2\pi^2k_Bm^*_{\textrm{therm}}T}{e\hbar B\cos{\theta}} \\
\label{Eqn:2:TempTermOscillationAmp}
\end{align}

\subsection{Mapping the Fermi surface}

\subsection{Measuring the spin mass}

\subsection{Measuring the band mass}

\subsubsection{Basic LK formula fitting}

Of all the damping terms in the \LK equation, only $R_T$ has any kind of temperature dependancy. This term also features the effective mass. By measuring oscillations at a fixed angle but with varying temperatures, the effective mass can be determined. However there is a difficulty in that in-order to observe oscillations, it is necessary to sweep the magnetic field, and many other damping terms have a field dependancy. To the first approximation, an inversely averaged applied field can be used in the \LK equation provided that the field sweep range is small. However there are a couple of techniques that were employed in order to overcome this shortcoming.

\subsubsection{Retrofitting ansatz LK formulae}
\label{Sec:2:LKRetrofitting}

One of the primary field-dependant contributions to the oscillation amplitude is the Dingle term scattering (equation \ref{Eqn:2:DingleTermOscillationAmp}). This has an exponential dependance with temperature. The Dingle factor, $\alpha$, can be determined by fitting a simplified version of equation \ref{Eqn:2:OscillationAmp} to oscillations which have been filtered to reduce the number of fitting parameters. Once we have the Dingle term, a series of ansatz data TODO...




\subsubsection{`Microfitting' the LK formula}
\label{Sec:2:LKMicrofitting}

\subsection{Method}

TODO: Angle correction

TODO: Temperature correction

\subsection{Yellow Magnet}

dHvA Measurements were all performed in Bristol on the `Yellow Magnet' system which was built by Oxford and can nominally operate up to \unit[20]{t} with use of the lambda plate although more typically is operated up to \unit[18]{T}. The sample sites on a one axis rotator and angle is determined by one of two orthogonal pick-up coils mounted on the sample stage weak, oscillating magnetic, field in

