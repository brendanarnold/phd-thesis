\section{A Comparison of Fermiology Techniques}
\label{Appendix:FermiologyTechniques}

\begin{table}
    \begin{center}
%        \caption{A summary of Fermiology techniques}
        \begin{tabular}[htbp]{llll}
\toprule
    & \ac{dHvA} & \ac{ARPES}    & Positron anihilation \\
\midrule
Regime  & Low (<\unit[3]{k}) Temperature and high ($0$--\unit[60]{T}) mangetic field required    & Any temperature possible, no magnetic field possible   & Any temperature possible, none to small magnetic field ($0$--\unit[0.7]{T}) may be used to guide incoming positrons \\
Resolution  & High, limited by temperature and crystal quality (at \unit[0.3]{K} smearing due to Fermi distribution)& $\sim \unit[2]{meV}$ and $\sim 0.3\%$ of B.Z.  & $\sim 10$--\unit[15]{\%} of B.Z. \\
Crystal & Requires high quality due to scattering, single crystal, may be very small ($< \unit[50]{\mu m}$)   & Quality not so important, course (i.e. greater than beam size) polycrystalline possible, needs to be large enough for beam   & Requires high quality due to positron pinning, single crystal, needs to be large enough for beam ($\sim \unit[1]{mm^2}$) \\
Superconductivity   & Cannot measure in the superconducting state   & Possible to measure in the superconudcting state  & Possible to measure in the superconducting state \\
Measurement region   & Bulk sensitive    & Surface sensitive ($\sim 0.5$--\unit[10]{nm})  & Bulk sensitive \\
Other notes & Does not locate the electron orbits within the Brillouin zone and does not map Fermi surface directly, as a result requires secondary knowledge such as \ac{DFT} calculations to back up data & Requires clean surfaces (i.e. ultra high vacuum, in-situ cleaving if possible) and  the $k_z$ direction is not easily accessible  &  Sensitive to open spaces within crystals, i.e. avoids high density regions of electrons such as CuO layers in cuprates    \\
\bottomrule
        \label{Table:Appendix:FermiologyTechniques}
        \end{tabular}
    \end{center}
\end{table}

