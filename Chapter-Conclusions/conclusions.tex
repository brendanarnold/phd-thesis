
In chapter~\ref{Sec:ResD} \ac{dHvA} measurements were presented on high quality samples of \BaFeP. Energy dispersions from \ac{DFT} calculations were tweaked to match the measured Fermi surface orbits by rigidly shifting both the inner and outer electron energy bands and the inner hole band by \unit{0.0050}{\textrm{Ry}}, \unit{0.0043}{\textrm{Ry}} and \unit{-0.0083}{\textrm{Ry}} respectively. The hourglass shaped outer hole band energy required no shift at the wide part and a shift of \unit{-0.0038}{\textrm{Ry}} at the narrow part which was found to be nested with the inner electron band. To achieve a smooth transition between the two energy shift regimes, the dispersion was tweaked proportionally to the \DzTwo{} band character and a complete Fermi surface was determined. Similar energy shifts were necessary to correct \ac{DFT} data for the closely related compound SrFe$_2$P$_2$ which is similarly nested but no shifts were necessary for the \ac{DFT} dispersions modelling CaFe$_2$P$_2$ which is not nested. The fact that electron-electron correlations are only accounted for in \ac{DFT} calculations on a mean-field level and that the discrepancies occur at areas which show nesting suggests that these shortcomings in the \ac{DFT} data are due to spin fluctuations arising from the nesting condition renormalising the band structure. To further investigate this the bare Lindhard susceptibility was computed and the nesting condition was clearly shown along the $q = (\pi, \pi)$ vector between hole and electron bands despite the fact that no superconductivity is present in \BaFeP. The conclusion is that partial nesting is not alone a sufficient condition for superconductivity.

The thermal effective masses were determined on the electron and, for the first time, hole orbits and the spin effective masses on the electron orbits. The masses showed a moderate renormalisation (between $\unit{0.88}{m_b}$ and $\unit{3.04}{m_b}$) on both hole and electron bands in line with previous literature~\cite{Shishido2010}.  Also presented in chapter~\ref{Sec:ResD} is an analytical fit to the Fermi surface similar to that in Bergemann \etal~\cite{Bergemann2000} for future use in theoretical models. It also includes a mapping of the progression of key orbit area sizes of the Fermi surface determined across the \BaFeAsP{} series using Vegard's law in order to aid the correction of \ac{DFT} calculations.

The next step in this line of research would be to continue performing \ac{dHvA} measurements on other members of the \BaFeAsP{} series in order to determine the topology and scale of the missing hole pockets, especially around the superconducting dome. This would provide further insight into whether nesting really is a crucial component of the exotic superconudctivity and also shed some light on the validity of \ac{DFT} calculations.

In chapter~\ref{Sec:HallBSCO} Hall measurements taken in high field from \unit{1.4}{\kelvin} to \unit{300}{\kelvin} on good quality samples of \ac{BSCO} were presented. A sharp change in $R_H(\unit{0}{\kelvin})/R_H(\unit{300}{\kelvin})$ was observed at $p=0.19$ which coincides with other phenomena which indicate the presence of the pseudogap~\cite{Tallon2001}. Since this occurs inside of the superconducting dome this is evidence in support of the scenario where the pseudogap drops inside of the superconducting dome.

A simple anisotropic scattering model based on the Ong construction~\cite{Ong1991} was fitted to the Hall data and the resulting scattering terms, $\Gamma = \Gamma_0 + \Gamma_1 \cos(2\phi)^2 T + \Gamma_2 T^2$ were used to calculate the longitudinal resistivity and normalised to the \unit{300}{\kelvin} values of the fitted resistivity curves. In general the comparison between the fitted resistivity curve coefficients and the coefficients obtained from the anisotropic model agreed within at least an order of magnitude with the linear in $T$ term matching to within a factor of 0.6 to 1.5. The residual $\rho^{R_H}_0$ term obtained from the model is consistently undervalued by a factor of around 2 to 10. The $T^2$ term from the anisotropic model is overvalued by up to a factor of 5.9  on the overdoped side to being undervalued by a factor of 5 on the underdoped side. This is likely due to the relative proximity to the van-Hove singularity and the fact that the model did not include a $v_F$ term in the scattering rate. The results provide a good starting point for further refinement and possible full agreement between temperature dependent Hall and resistivity data in the cuprates without resorting to complicated Fermi surface reconstruction models such as that proposed by LeBoeuf \etal~\cite{LeBoeuf2011}.

Finally, presented in chapter~\ref{Sec:HallBSCO} is a novel doping determination technique based on a method of matching high temperature Hall coefficient to \ac{TL2201} as a reference material. The resulting dopings are greater than those from the `universal' Presland/Tallon relation~\cite{Presland1991} and less than those assigned to similar samples measured with \ac{ARPES}~\cite{Kondo2004}. The overall spread in dopings for these samples from this new method was determined to be between $p=0.12$ and $p=0.36$.

Further refinements on the anisotropic scattering model is required to provide solid evidence that the model is an appropriate explanation of the temperature dependence of the low temperature Hall behaviour. In particular the effects of the Fermi velocity on the model close to the van-Hove singularity can be included relatively easily and would show whether this is the cause of the disparity of the model at higher temperatures. Moreover, a more precise determination of the low temperature Hall behaviour in \ac{BSCO} would better constrain the model as well as resolve any ambiguities as to whether the low temperature behaviour is, for example, linear in temperature.
