
In chapter~\ref{Sec:ResD} \ac{dHvA} measurements were presented on high quality samples of \BaFeP. Energy dispersions from \ac{DFT} calculations were tweaked to match the measured Fermi surface orbits by rigidly shifting both the inner and outer electron energy bands and the inner hole band by \unit{0.0050}{\textrm{Ry}}, \unit{0.0043}{\textrm{Ry}} and \unit{-0.0083}{\textrm{Ry}} respectively. The hourglass shaped outer hole band energy required no shift at the wide part and a shift of \unit{-0.0038}{\textrm{Ry}} at the narrow part which was found to be nested with the inner electron band. To achieve a smooth transition between the two energy shift regimes, the dispersion was tweaked proportionally to the \DzTwo band character and a complete Fermi surface was determined. Similar energy shifts were necessary to correct \ac{DFT} data for the closely related compound SrFe$_2$P$_2$ which is similarly nested but no shifts were necessary for the \ac{DFT} dispersions modelling CaFe$_2$P$_2$ which is not nested. The fact that electron-electron correlations are only accounted for in \ac{DFT} calculations on a mean-field level and that the discrepancies occur at areas which show nesting suggests that these shortcomings in the \ac{DFT} data are due to spin fluctuations arising from the nesting condition renormalising the band structure. To further investigate this the bare Lindhard susceptibility was computed and the nesting condition was clearly shown along the $q = (\pi, \pi)$ vector between hole and electron bands despite the fact that no superconductivity is present in \BaFeP. The conclusion is that partial nesting is not alone a sufficient condition for superconductivity.

The thermal effective masses were determined on the electron and, for the first time, hole orbits and the spin effective masses on the electron orbits. The masses showed a moderate renormalisation (between $\unit{0.88}{m_b}$ and $\unit{3.04}{m_b}$) on both hole and electron bands in line with previous literature~\cite{Shishido2010}.  Also presented in chapter~\ref{Sec:ResD} is an analytical fit to the Fermi surface similar to that in Bergemann \etal~\cite{Bergemann2000} for future use in theoretical models. It also includes a mapping of the progression of key orbit area sizes of the Fermi surface determined across the \BaFePAs series using Vegard's law in order to aid the correction of \ac{DFT} calculations.

In chapter~\ref{Sec:HallBSCO} Hall measurements taken in high field from \unit{1.4}{\kelvin} to \unit{300}{\kelvin} on good quality samples of \ac{BSCO} were presented. A sharp change in $R_H(\unit{0}{\kelvin})/R_H(\unit{300}{\kelvin})$ was observed at $p=0.19$ which coincides with ... \TODO{Ask Nigel what his take on it was}.

A simple model based on the Ong construction~\cite{Ong1991} was fitted to the Hall data and relative magnitudes of the scattering terms, $\Gamma = \Gamma_0 + \Gamma_1 \cos(2\phi)^2 T + \Gamma_2 T^2$ were compared with terms obtained from fits to resistivity curves. The underdoped samples were found to agree within a factor of around $\times 2.5$ although the overdoped samples only agreed within one or two orders of magnitude. This is likely due to the relative proximity to the van-Hove singularity and the fact that the model did not include a $v_F$ term in the scattering rate. The results provide a good starting point for further refinement and possible full agreement between temperature dependent Hall and resistivity data in the cuprates without resorting to complicated Fermi surface reconstruction models such as that proposed by LeBoeuf \etal~\cite{LeBoeuf2011}.

Finally, presented in chapter~\ref{sec:HallBSCO} is a novel doping determination technique based on a method of matching high temperature Hall coefficient to \ac{TL2201} as a reference material. The resulting dopings are greater than those from the `universal' Presland/Tallon relation~\cite{Presland1991} and less than those assigned to similar samples measured with \ac{ARPES}~\cite{Kondo2004}. The overall spread in dopings for these samples from this new method was determined to be between $p=0.12$ and $p=0.36$.
