
\section{Discussion}

Similar differences between the measured data and calculations were observed for the sister 122 compound SrFe$_2$P$_2$\cite{Analytis2009} and is entirely consistent with results obtained on the entire \BaFePAs series\cite{Shishido2010}. Notably however, the non-nested 122 pnictide compound CaFe$_2$P$_2$ does not show any such differences to the DFT calculations, suggesting that perhaps the shifts in energy may arise from spin-fluctuations.

Orbital character is negligible for s and p orbitals for all bands which confirms ...

Energy shifts for band $2$ are proprtional to the \DzTwo and \DxzDyz characters, implies there may be a link between the \kz scattering and energy enhancements.


It is interesting to note that the DFT applied energy shifts apply to partially nested Fermi surfaces, whereas the large, unested portion of band 2 has zero shift. Other partially nested pnictide materials such as LaFePO\cite{Carrington2009} and SrFe$_{2}$P$_{2}$\cite{Analytis2009} have similar shifts whereas the non-nested, material CaFe$_{2}$P$_{2}$\cite{Coldea2009} matches the DFT calculations well without any rigid energy shift. This correlation between nesting and corrections to the DFT calculation lends weight to the notion that the shifts in the calculation energies are cause by spin fluctuations.


