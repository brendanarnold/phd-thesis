
\section{X-Ray Diffraction}
    \label{Sec:3:XrayDiffraction}

The crystalline axes of the sample were determined on a Kappa Apex II single crystal diffractometer with the aid of Mairi Haddow. The sample was mounted on a glass rod and held in place using vacuum grease. Clear diffraction spots are visible on the example scans shown in \fig\ref{Fig:3:XRayDiffraction} although there is some evidence of a second, misaligned phase with the doubling of the spots in a small number of the scans such as the one in the top left panel. There is further evidence of a secondary phase as some peaks are doubled up in the dHvA data presented later. However it appears that the affects are minor and do not influence the overall measurements and conclusions in any appreciable way.
%%
\begin{figure}[htbp]
    \begin{center}
        \includegraphics[scale=0.7]{Chapter3-dHvABaFe2P2/Figures/Xrays/XRayDiffraction/XRayDiffraction}
        \caption{Top left, top right and bottom left show example diffraction patterns of the \BaFeP sampple. Top-left shows a zoomed portion of doubled peaks indicating that there may potentially be a misalignment within the crystal. Bottom right shows the labeled crystal axes superimposed on the sample which is mounted on a glass rod.}
        \label{Fig:3:XRayDiffraction}
    \end{center}
\end{figure}

Lattice parameters are determined using the Apex II software and are presented in table \ref{Table:3:LatticeParams} along with comparisons to two previous measurements found in the literature. The result agree within the error.
%%
\medskip
%% 
\begin{table}[htbp]
    \begin{center}
        \caption{Lattice paramters from XRD measurements compared with literature.}
        \begin{tabular}{lrrr}
\toprule
Source  &  $a$ (\AA) & $c$ (\AA) & $z_P$ (\% $c$)\\
\midrule
X-ray   & $3.86(4)$  & $12.42(9)$ & \\
Rotter et al.\cite{Rotter2010} & $3.8435(4) $ & $12.422(2)$ & $34.59(1)$ \\
Mewis et al.\cite{Mewis1980} & $3.63$ & $11.76$ & $34.56$ \\
\bottomrule
        \label{Table:3:LatticeParams}
        \end{tabular}
    \end{center}
\end{table}

